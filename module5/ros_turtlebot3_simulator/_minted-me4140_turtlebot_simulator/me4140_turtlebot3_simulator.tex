\documentclass[12pt]{article}

\usepackage{minted}
%\newminted[python]{python}{frame=single}
%\fvset{showspaces}
%\renewcommand\FancyVerbSpace{\textcolor{mygray}{\char32}}
\setminted[text]{
escapeinside=||, 
%breaksymbolleft=\carriagereturn,
frame=single,
%showspaces=true
framesep=2mm,
baselinestretch=1.2,
bgcolor=mygray
}
\setminted[cpp]{
escapeinside=||, 
%breaksymbolleft=\carriagereturn,
frame=single,
%showspaces=true
framesep=2mm,
baselinestretch=1.2,
bgcolor=mygray
}
\usepackage{xcolor}
\usepackage{hyperref}
\usepackage[pdftex]{graphicx}
\usepackage{multirow}
\usepackage{setspace}
\usepackage{color}
\usepackage{multicol}
\usepackage{listings}
\usepackage{color}
\usepackage[T1]{fontenc}

\hypersetup{
    bookmarks=true,         % show bookmarks bar?
    unicode=false,          % non-Latin characters in Acrobat’s bookmarks
    pdftoolbar=true,        % show Acrobat’s toolbar?
    pdfmenubar=true,        % show Acrobat’s menu?
    pdffitwindow=false,     % window fit to page when opened
    pdfstartview={FitH},    % fits the width of the page to the window
    pdftitle={My title},    % title
    pdfauthor={Author},     % author
    pdfsubject={Subject},   % subject of the document
    pdfcreator={Creator},   % creator of the document
    pdfproducer={Producer}, % producer of the document
    pdfkeywords={keyword1, key2, key3}, % list of keywords
    pdfnewwindow=true,      % links in new PDF window
    colorlinks=true,       % false: boxed links; true: colored links
    linkcolor=red,          % color of internal links (change box color with linkbordercolor)
    citecolor=green,        % color of links to bibliography
    filecolor=magenta,      % color of file links
    urlcolor=blue           % color of external links
}

\definecolor{dkgreen}{rgb}{0,0.6,0}
\definecolor{gray}{rgb}{0.5,0.5,0.5}
\definecolor{mauve}{rgb}{0.58,0,0.82}
\lstset{frame=tb,
  language=Java,
  aboveskip=3mm,
  belowskip=3mm,
  showstringspaces=false,
  columns=flexible,
  basicstyle={\small\ttfamily},
  numbers=none,
  numberstyle=\tiny\color{gray},
  keywordstyle=\color{blue},
  commentstyle=\color{dkgreen},
  stringstyle=\color{mauve},
  breaklines=true,
  breakatwhitespace=true,
  tabsize=3
}

% ME4140 - Fall 2016 - Fall 2017 - Fall 2019



\textwidth=6.5in
\topmargin=-0.5in
\textheight=9.25in
\hoffset=-0.5in
\footskip=0.2in

\pagestyle{myheadings}
\markright{{\large ME 4140 Fall 2019---The Robotic Operating System}}

\definecolor{dkgreen}{rgb}{0,0.6,0}
\definecolor{gray}{rgb}{0.5,0.5,0.5}
\definecolor{mauve}{rgb}{0.58,0,0.82}

\definecolor{mygray}{rgb}{.6, .6, .6}
\definecolor{mypurple}{rgb}{0.6,0.1961,0.8}
\definecolor{mybrown}{rgb}{0.5451,0.2706,0.0745}
\definecolor{mygreen}{rgb}{0, .39, 0}

\newcommand{\R}{\color{red}}
\newcommand{\B}{\color{blue}}
\newcommand{\BR}{\color{mybrown}}
\newcommand{\K}{\color{black}}
\newcommand{\G}{\color{mygreen}}
\newcommand{\PR}{\color{mypurple}}

\newcommand{\pkgname}{<package\_name>}
\newcommand{\wspname}{<workspace\_name>}
\newcommand{\nodname}{<node\_name>}
\newcommand{\tpcname}{<topic\_name>}
\newcommand{\lfname}{<file\_name>}

\newcommand{\home}{\textasciitilde/}

\newcommand{\rosdistro}{melodic}

\newcommand{\pthname}{/opt/ros/\rosdistro/share/turtlebot\_stage/maps/}


\begin{document}

\thispagestyle{plain}

\begin{center}
   {\bf \Large ROS - The Turtlebot Simulator}\vspace{2mm} \\
   {\bf \large ME 4140 - Introduction to Robotics - Fall 2019} \\
\end{center}

\includegraphics[scale=.5]{tb3.png}

\begin{enumerate}
	\item Update your linux system before you get started. 
	\begin{minted}{text} 
	sudo apt-get update
	\end{minted}

	\item Install the necessary nodes into your ROS system. This tutorial comes from \href{http://emanual.robotis.com/docs/en/platform/turtlebot3/simulation/#simulation} {here.} \\
    	
    	{\bf turtlebot3 }
	\begin{minted}{text} 
	sudo apt-get install ros-|\rosdistro|-turtlebot3
	\end{minted}
 	 {\bf turtlebot3\_simulations}
	\begin{minted}{text} 
	sudo apt-get install ros-|\rosdistro|-turtlebot3-simulations
	\end{minted}
 	{\bf turtlebot3\_gazebo}
	\begin{minted}{text} 
	sudo apt-get install ros-|\rosdistro|-turtlebot3-gazebo
	\end{minted}


%    \item Next install the physical 'turtlebot' drivers into your ROS system. This step is only necessary if you are using a real turtlebot. \href{http://wiki.ros.org/Robots/TurtleBot} {Link Here} 
%   \begin{minted}{text}  
%(sudo apt-get install ros-|\rosdistro|-turtlebot ros-|\rosdistro|-turtlebot-apps
%ros-|\rosdistro|-turtlebot-interactions ros-|\rosdistro|-turtlebot-simulator 
%ros-|\rosdistro|-kobuki-ftdi ros-|\rosdistro|-rocon-remocon 
%ros-|\rosdistro|-rocon-qt-library ros-|\rosdistro|-ar-track-alvar-msgs})
%\end{minted}
%    
    \item Test the simulator. First set the environment variable TURTLEBOT\_MODEL. Add this line to the .bashrc script so you do not have to do it for each terminal. \\
	\begin{minted}{text} 
	export TURTLEBOT3_MODEL=burger
	\end{minted}
	Then turn on the simulator. 
	\begin{minted}{text} 
	roslaunch turtlebot3_gazebo turtlebot3_world.launch
	\end{minted}

	You should see the gazebo window open containing your robot. Test that the keyboard drives the robot. This may take some time.
	\begin{minted}{text} 
	roslaunch turtlebot3_teleop turtlebot3_teleop_key.launch
	\end{minted}
	    
	    \item Now turn on the node to produce robot data in the simulated world.  \\

	\begin{minted}{text} 
	roslaunch turtlebot3_gazebo turtlebot3_simulation.launch
	\end{minted}

	Open RVIZ to view the data. This is a very useful tool. 	
	\begin{minted}{text} 
	roslaunch turtlebot3_gazebo turtlebot3_gazebo_rviz.launch
	\end{minted}
	
	\item Next we are going to learn about SLAM and GMAPPING ! Please see the tutorial referenced above if you are ready to proceed.\\
%    \begin{itemize}
%    
%        \item {\fontfamily{qcr}\selectfont  \hspace{5mm} \pthname maze.png}
%        \item {\fontfamily{qcr}\selectfont  \hspace{5mm} \pthname maze.yaml}
%        \item {\fontfamily{qcr}\selectfont  \hspace{5mm} \pthname stage/maze.world}
%    
%    \end{itemize}

%    \item First try the simulator in the demo world called {\it maze}. We will export the files as {\it environment variables}
%
%    {\fontfamily{qcr}\selectfont  \hspace{5mm} \$ export TURTLEBOT\_STAGE\_MAP\_FILE=\\"\pthname maze.yaml"}\\
% 
%    {\fontfamily{qcr}\selectfont  \hspace{5mm} \$ export TURTLEBOT\_STAGE\_WORLD\_FILE=\\"\pthname stage/maze.world"}\\
%    
%    \item Now use the launch file (available upon install) to start the simulator.\\
%    {\fontfamily{qcr}\selectfont  \hspace{5mm} \$ roslaunch turtlebot\_stage turtlebot\_in\_stage.launch}
%    
%    \item Now you can modify the world you have just simulated. To do this copy all three files and rename them something sensible. Open the {\it .png} file with any image editor, and draw on it and save. You also need to modify just a few lines in the {\it .yaml} file and the {\it .world} file. (Note: This step will be detailed in the next tutorial. Continue at your own risk or contact me for help.)
%    
%     \item Did you notice an error when you turned the node on? We can fix that.  \\\\
%    
%    	 {\fontfamily{qcr}\selectfont  \hspace{5mm} \$ sudo  gedit /opt/ros/\rosdistro/share/gmapping/nodelet\_plugins.xml}\\\\
%    	 
%    	 Copy the code below into the new file. This a bug related to moving to `kinetic'.\\
%    \lstset{language=XML}
%     \begin{lstlisting}
%
%<library path="lib/libslam_gmapping_nodelet">
%    <class name="SlamGMappingNodelet" type="SlamGMappingNodelet" base_class_type="nodelet::Nodelet">
%        <description>
%            Nodelet ROS wrapper for OpenSlams Gmapping.
%        </description>
%    </class>
%</library>
%      \end{lstlisting}
%
%\vspace{5mm}    Now run your node again.
\end{enumerate}
\end{document}

