\documentclass[12pt]{article}
\usepackage[pdftex]{graphicx}
\usepackage{multicol}
\usepackage{multirow}
\usepackage{setspace}
\usepackage{color}
\usepackage{listings}
\usepackage[T1]{fontenc}
\usepackage[margin=1in]{geometry}
\usepackage{color,soul}
\usepackage{fancyvrb}
\usepackage{framed}
\usepackage{wasysym}
\usepackage{array}
\usepackage{hyperref}


\usepackage{listings}

\usepackage{textcomp}

\usepackage[T1]{fontenc} 

\hypersetup{
    bookmarks=true,         % show bookmarks bar?
    unicode=false,          % non-Latin characters in Acrobat’s bookmarks
    pdftoolbar=true,        % show Acrobat’s toolbar?
    pdfmenubar=true,        % show Acrobat’s menu?
    pdffitwindow=false,     % window fit to page when opened
    pdfstartview={FitH},    % fits the width of the page to the window
    pdftitle={My title},    % title
    pdfauthor={Author},     % author
    pdfsubject={Subject},   % subject of the document
    pdfcreator={Creator},   % creator of the document
    pdfproducer={Producer}, % producer of the document
    pdfkeywords={keyword1, key2, key3}, % list of keywords
    pdfnewwindow=true,      % links in new PDF window
    colorlinks=true,       % false: boxed links; true: colored links
    linkcolor=red,          % color of internal links (change box color with linkbordercolor)
    citecolor=green,        % color of links to bibliography
    filecolor=magenta,      % color of file links
    urlcolor=blue           % color of external links
}

\definecolor{dkgreen}{rgb}{0,0.6,0}
\definecolor{gray}{rgb}{0.5,0.5,0.5}
\definecolor{mauve}{rgb}{0.58,0,0.82}

\definecolor{mygray}{rgb}{.6, .6, .6}
\definecolor{mypurple}{rgb}{0.6,0.1961,0.8}
\definecolor{mybrown}{rgb}{0.5451,0.2706,0.0745}
\definecolor{mygreen}{rgb}{0, .39, 0}

\newcommand{\R}{\color{red}}
\newcommand{\B}{\color{blue}}
\newcommand{\BR}{\color{mybrown}}
\newcommand{\K}{\color{black}}
\newcommand{\G}{\color{mygreen}}
\newcommand{\PR}{\color{mypurple}}



\setulcolor{red} 
\setstcolor{green} 
\sethlcolor{mygray} 

\lstset{frame=tb,
  language=Java,
  aboveskip=3mm,
  belowskip=3mm,
  showstringspaces=false,
  columns=flexible,
  basicstyle={\small\ttfamily},
  numbers=none,
  numberstyle=\tiny\color{gray},
  keywordstyle=\color{blue},
  commentstyle=\color{dkgreen},
  stringstyle=\color{mauve},
  breaklines=true,
  breakatwhitespace=true,
  tabsize=3
}

% ME4140 - Fall 2016

\textwidth=6.5in
\topmargin=-0.5in
\textheight=9.25in
\hoffset=-0.5in
\footskip=0.2in

\pagestyle{myheadings}
\markright{{\large ME 4140 Fall 2019---The Robotic Operating System}}

\newcommand{\pkgname}{\G<package\_name>\K}
\newcommand{\wspname}{\R<workspace\_name>\K}
\newcommand{\nodname}{\PR<node\_name>\K}
\newcommand{\tpcname}{/topic\_name}
\newcommand{\home}{\textasciitilde/}

\newcommand{\rosdistro}{kinetic}

\begin{document}

\thispagestyle{plain}

\begin{center}
   {\bf \Large ROS - Topics, Publishing and Subscribing}\vspace{2mm} \\
   {\bf \large ME 4140 - Introduction to Robotics - Fall 2019} \vspace{5mm}\\
\end{center}

\begin{enumerate}
    \item  \href{http://wiki.ros.org/Topics}{\bf Topics:} ROS nodes can communicate by {\bf publishing} and {\bf subscribing} to topics. A topic is information generated by a publishing node that is made available to a subscribing node or nodes in the ROS system.
        
        \begin{itemize}		
            \item A node can publish a topic. This node is a publisher.    	
            \item A node can subscribe to a topic. This node is a subscriber.
            \item Most nodes publish and subscribe to multiple topics. 
            \item Integrate built-in ROS nodes and modify your own ROS nodes in C++, Python, and even MATLAB
            
        \end{itemize}

	\item \href{http://wiki.ros.org/catkin/Tutorials/create_a_workspace}{\bf Setup the Workspace:} To our our own ROS nodes we need to setup a {\it catkin workspace}. Catkin is the program that manages the file system behind the scenes. It is our working directory or environment in which we can customize our ROS system.

	\begin{description}

		\item[Step 1:] Source the installation files needed to create a workspace. This requires ROS to be previously installed.\\\\
		{\bf \texttt{\$ source /opt/ros/\rosdistro/setup.bash}}\\

%		\item[Step 2:] Open a new terminal and navigate to the future location of your workspace.\\\\
%		{\fontfamily{qcr}\selectfont  \hspace{5mm} \$ cd $\sim$} \hspace{10mm}This is a shortcut for: {\fontfamily{qcr}\selectfont  \hspace{5mm} \$ cd 
% home/<username>}\\

		\item[Step 2:] Choose a workspace name and create a workspace and source directory with {\it mkdir}. This step determines the location of your new workspace.\\\\
{\bf \texttt{\$ mkdir -p \home\wspname/src}}\\
	\item[Step 3:] Navigate to the top of your workspace directory and build your workspace.\\\\
{\bf \texttt{\$ cd \home\wspname} }\hspace{10mm}or try this {\bf \texttt{ \$ cd ..}}\\\\	
{\bf \texttt{\$ catkin\_make}}\\
	\item [Step 4:]Before continuing test that your ROS system is setup correctly.\\\\
{\bf \texttt{\$ source devel/setup.bash}}\\\\
{\bf \texttt{\$ echo \$ROS\_PACKAGE\_PATH}}\\

You should see somthing like this in the terminal. This is where ROS is installed.	\\	\\
{\bf \texttt{\$ /home/<user\_name>/\wspname/src:/opt/ros/\rosdistro/share}}

	\end{description}

\newpage
    \item \href{http://wiki.ros.org/ROS/Tutorials/WritingPublisherSubscriber(c++)}{{\bf Create a Node:}} You can write custom nodes for your ROS system in C++, Python, or Lisp. These documents will support C++.
         \begin{description}    				
            \item[Step 1:] \href{http://wiki.ros.org/ROS/Tutorials/CreatingPackage}{Create a new package} in your workspace for your new node to belong to. Make sure to do this in the correct directory   \\  \\   
            {\bf \texttt{\$ cd \home\wspname/src }}\\\\
            {\bf \texttt{\$ catkin\_create\_pkg \pkgname\hspace{3mm}std\_msgs rospy roscpp }}\\
            
            
            
            \item[Step 2:] Back out to the workspace directory then compile your package with \href{http://wiki.ros.org/catkin/Tutorials/using_a_workspace#Building_Packages_in_a_catkin_Workspace}{catkin\_make} \\\\
            {\bf \texttt{\$ cd \home\wspname}}\\\\
            {\bf \texttt{\$ catkin\_make}}\\
            
			\item Now you need to source the workspace directory.\\\\
			{\bf \texttt{\$ source \home\wspname/devel/setup.bash}}\\\\

			\item Also Modify your {\bf .bashrc} file so that this happens every time you start the terminal.\\\\
			{\bf \texttt{\$ gedit \home.bashrc}}\\\\
			A text editor will open. You may see the following line to the bottom of the file. If it is not there add it on the bottom line. Save and close the file.\\\\
			{\bf \texttt{\$ source \home\wspname/devel/setup.bash}}\\\\
					
            If you get here with no errors you are ready to write some code and test your new package!
            
     
  \newpage
  \item[Step 3:] Write the {\bf publisher node} in C++. It will start as C++ code and then it will be compiled into an executable. Create a file from the sample code {\bf ttu\_publisher.cpp} It needs to be saved in the correct directory.\\\\
 {\bf \texttt{\$ gedit \home\wspname/src/\pkgname/src/\nodname.cpp}}. \\\\
  Copy the code below into the text editor. It should be saved as a \nodname.cpp \\
  
            \begin{lstlisting}

#include "ros/ros.h"
#include "geometry_msgs/Twist.h"
#include <sstream>

int main(int argc, char **argv)
{
    ros::init(argc, argv, "node_name");

    ros::NodeHandle n;
    ros::Publisher ttu_publisher = n.advertise<geometry_msgs::Twist>("/turtle1/cmd_vel", 1000);
    ros::Rate loop_rate(10);

    int count = 0;
    while (ros::ok())
    {
        geometry_msgs::Twist msg;
        msg.linear.x = 2+0.01*count;
        msg.angular.z = 2;
        ttu_publisher.publish(msg);
        ros::spinOnce();
        loop_rate.sleep();
        count++;
    }
}
\end{lstlisting}

\item[Step 4:] Before we can compile the node we have to modify the file below.\\

{\bf \texttt{\$ gedit \home\wspname/src/\pkgname/CMakeLists.txt}} \\ 

Add the following lines to the bottom of the file and save. \\

{\bf \texttt{\$add\_executable(\nodname\hspace{3mm}src/\nodname.cpp) }} \\
{\bf \texttt{\$ target\_link\_libraries(\nodname \hspace{3mm}\$\{catkin\_LIBRARIES\}) }} \\
{\bf \texttt{\$ add\_dependencies(\nodname \hspace{3mm}beginner\_tutorials\_generate\_messages\_cpp) }}\\\\
 
% \newpage
 

\item[Step 5:] Test and compile the new publisher node.

This will build your system and check for errors in your entire workspace.\\\\
{\bf \texttt{\$ cd \home\wspname}} \\
{\bf \texttt{\$ catkin\textunderscore make}} \\\\
Start a core\\\\
{\bf \texttt{\$ roscore}}\\\\
Start your new node\\\\
{\bf \texttt{\$ rosrun \pkgname\hspace{3mm}\nodname}}\\\\
{\bf \texttt{\$ rostopic list}}\\\\
You should now be able to see your topic. Now lets do something more fun. Turn on a turtle.\\\\

{\bf \texttt{\$ rosrun turtlesim turtlesim\textunderscore node}}\\\\
Now start your publisher node with the cmd\textunderscore vel topic patched through to the turtle like we did previously.\\\\
{\bf \texttt{\$ rosrun \pkgname\hspace{3mm}\nodname \hspace{3mm}\tpcname:=/turtle1/cmd\textunderscore vel}}\\\\
 \end{description}
 
\newpage

\item Now create a {\bf subscriber node} in the same package as the previous node. You can follow the tutorial \href{http://wiki.ros.org/ROS/Tutorials/WritingPublisherSubscriber(c++)} {here}. 

\begin{description}
\item [Step 1:] This time use the file below called {\bf ttu\_subscriber.cpp} to start.\\

  \begin{lstlisting}

#include "ros/ros.h"
#include "std_msgs/String.h"
#include "geometry_msgs/Twist.h"
/**
* This tutorial demonstrates simple receipt of messages over the ROS system.
*/
void dataCallback(const geometry_msgs::Twist::ConstPtr& msg)
{
	ROS_INFO("I heard: [%f]", msg->linear.x);
}
int main(int argc, char **argv)
{
	ros::init(argc, argv, "ttu_subscriber");
	ros::NodeHandle n;
	ros::Subscriber sub = n.subscribe("/cmd_vel", 1000, dataCallback);
	ros::spin();
	return 0;
}


\end{lstlisting}


\item [Step 2:] Modify the CMakeLists.txt file as you did previously.\\\\
{\bf \texttt{\$\hspace{5mm}\$ gedit \home\wspname/src/\pkgname/CMakeLists.txt }}

\item [Step 3:] Compile the new subscriber node.\\\\
{\bf \texttt{\$ cd \home\wspname}}\\
{\bf \texttt{\$ catkin\_make}}\\

\item [Step 4:] Test the new node. Does it work?\\

{\bf \texttt{\$ rosrun \pkgname\hspace{3mm}\nodname }}\\\\

\end{description}
\vspace*{20mm}
\item Bored with all of that? Try this \href{http://wiki.ros.org/joy/Tutorials/WritingTeleopNode}{JoyStick Teleop Node} for use with a Linux compatable joystick.\\ 

\end{enumerate}
\end{document}

