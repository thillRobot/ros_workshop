% !TEX TS-program = xelatex
% !TEX encoding = UTF-8 Unicode

% Tennessee Technological University
% ME4140 - Fall 2016 - Fall 2017 - ? - Fall 2019 - Fall 2020
% Tristan Hill - September 19, 2020 - October 09, 2020
% Turtorial 5 - Turtlebot3 Simulator

\documentclass[12pt]{article}

% Custom Preamble
\usepackage{/home/thill/Documents/lectures/ros_lectures/ros_tutorial} 

% Title and Misc
\newcommand{\MNUM}{5} %Module Number
\newcommand{\MNAME}{Turtlebot3 Simulator} %Module Name
\pagestyle{myheadings}
\markright{{\large ME4140 - ROS Workshop - Fall 2020}}

\begin{document}

\thispagestyle{plain}

\begin{center}
   {\bf \Large ROS Workshop - Tutorial\hspc\MNUM\hspc - \MNAME}\vspace{3mm}\\
   {\bf \large ME 4140 - Introduction to Robotics - Fall 2020} \vspace{5mm}\\
\end{center}

\includegraphics[scale=.5]{turtlebot3_family.png} \includegraphics[scale=.35]{turtlebot3_simulations.png}


\begin{description}[labelindent=1cm]
	
	\item[\textbf{\underline{Overview:}}] \hfill \vspace{3mm}\\
	After completing {\it Tutorial 4 - Create Package}, You have learned some basics of ROS, and you can are ready for a more advanced robot. You can read more \href{https://www.turtlebot.com/}{here} and \href{https://emanual.robotis.com/docs/en/platform/turtlebot3/overview/}{here}.
	
	\item[\textbf{\underline{System Requirements:}}] \hfill \vspace{0mm}

\begin{itemize}
	\item {\bf ROS+OS}: This tutorial is intended for a system with ROS Melodic installed on the Ubuntu 18.04 LTS operating system. Alternate versions of ROS (i.e. - Kinetic, Noetic, etc.) may work but have not been tested. Versions of ROS are tied to versions of Ubuntu.
	\item {\bf ROS:} Your computer must be connected to the internet to proceed. Update the system before you begin.
	\item {\bf Workspace Setup:} You must have successfully setup a Catkin Workspace in tutorial 4.  
\end{itemize}

	
	\item[\textbf{\underline{Disclaimer:}}] \hfill \vspace{0mm}
	
	\begin{itemize}

		\item {\R\underline{\bf Backup the System:}} If you are using a virtual machine, it is recommend to make a snaphot of your virtual machine before you start each module. In the event of an untraceable error, you can restore to a previous snapshot. 
		
		\item \underline{\B ROSLAUNCH:} This tutorial involves using the roslaunch command which runs a muliple of nodes at once as described in the launch file. We will learn more about this later. 
	
		 
	\end{itemize}


	



	\newpage 

\item[\textbf{\underline{Part 1 - Turtlebot3 Installation:}}] \hfill \vspace{0mm}

\begin{enumerate}
	\item Update your linux system before you get started. 
	\begin{minted}{text} 
	sudo apt update
	\end{minted}

	\item Install the necessary nodes into your ROS system. This tutorial comes from \href{http://emanual.robotis.com/docs/en/platform/turtlebot3/simulation/#simulation} {here.} 
    	
    	{\bf turtlebot3 }
	\begin{minted}{text} 
sudo apt install ros-|\rosdistro|-turtlebot3
	\end{minted}
 	 {\bf turtlebot3\_simulations}
	\begin{minted}{text} 
sudo apt install ros-|\rosdistro|-turtlebot3-simulations
	\end{minted}
{\bf turtlebot3\_gazebo}
	\begin{minted}{text} 
sudo apt install ros-|\rosdistro|-turtlebot3-gazebo
	\end{minted}


%    \item Next install the physical 'turtlebot' drivers into your ROS system. This step is only necessary if you are using a real turtlebot. \href{http://wiki.ros.org/Robots/TurtleBot} {Link Here} 
%   \begin{minted}{text}  
%(sudo apt install ros-|\rosdistro|-turtlebot ros-|\rosdistro|-turtlebot-apps
%ros-|\rosdistro|-turtlebot-interactions ros-|\rosdistro|-turtlebot-simulator 
%ros-|\rosdistro|-kobuki-ftdi ros-|\rosdistro|-rocon-remocon 
%ros-|\rosdistro|-rocon-qt-library ros-|\rosdistro|-ar-track-alvar-msgs})
%\end{minted}
%    
\end{enumerate}	
\newpage

\item[\textbf{\underline{Part 2 - Turtlebot3 Testdrive:}}] \hfill \vspace{0mm}
\begin{enumerate}
    \item Test the simulator. The environment variable TURTLEBOT\_MODEL must be set to choose your robot type. Use echo to add this line to the .bashrc script so you do not have to do it for each terminal. \\
	\begin{minted}{text} 
echo "export TURTLEBOT3_MODEL=burger" >> ~/.bashrc
	\end{minted}
	Then turn on the simulator. 
	\begin{minted}{text} 
roslaunch turtlebot3_gazebo turtlebot3_world.launch
	\end{minted}

	You should see the gazebo window open containing your robot. Test that the keyboard drives the robot. Enter the following command in a new terminal.
	\begin{minted}{text} 
roslaunch turtlebot3_teleop turtlebot3_teleop_key.launch
	\end{minted}
	    
	    \item Now turn on the node to produce robot data in the simulated world.  \\

	\begin{minted}{text} 
roslaunch turtlebot3_gazebo turtlebot3_simulation.launch
	\end{minted}

	Open RVIZ to view the data. This is a very useful tool. 	
	\begin{minted}{text} 
roslaunch turtlebot3_gazebo turtlebot3_gazebo_rviz.launch
	\end{minted}
\end{enumerate}

\item[\textbf{\underline{Tutorial Complete:}}] \hfill \vspace{3mm}\\ 
	After completing {\it Tutorial 5 - Turtlebot3 Simulator}, you are ready to learn about robot navigation with SLAM and GMAPPING ! Please see the tutorial referenced above if you are ready to proceed.\\
%    \begin{itemize}
%    
%        \item {\fontfamily{qcr}\selectfont  \hspace{5mm} \pthname maze.png}
%        \item {\fontfamily{qcr}\selectfont  \hspace{5mm} \pthname maze.yaml}
%        \item {\fontfamily{qcr}\selectfont  \hspace{5mm} \pthname stage/maze.world}
%    
%    \end{itemize}

%    \item First try the simulator in the demo world called {\it maze}. We will export the files as {\it environment variables}
%
%    {\fontfamily{qcr}\selectfont  \hspace{5mm} \$ export TURTLEBOT\_STAGE\_MAP\_FILE=\\"\pthname maze.yaml"}\\
% 
%    {\fontfamily{qcr}\selectfont  \hspace{5mm} \$ export TURTLEBOT\_STAGE\_WORLD\_FILE=\\"\pthname stage/maze.world"}\\
%    
%    \item Now use the launch file (available upon install) to start the simulator.\\
%    {\fontfamily{qcr}\selectfont  \hspace{5mm} \$ roslaunch turtlebot\_stage turtlebot\_in\_stage.launch}
%    
%    \item Now you can modify the world you have just simulated. To do this copy all three files and rename them something sensible. Open the {\it .png} file with any image editor, and draw on it and save. You also need to modify just a few lines in the {\it .yaml} file and the {\it .world} file. (Note: This step will be detailed in the next tutorial. Continue at your own risk or contact me for help.)
%    
%     \item Did you notice an error when you turned the node on? We can fix that.  \\\\
%    
%    	 {\fontfamily{qcr}\selectfont  \hspace{5mm} \$ sudo  gedit /opt/ros/\rosdistro/share/gmapping/nodelet\_plugins.xml}\\\\
%    	 
%    	 Copy the code below into the new file. This a bug related to moving to `kinetic'.\\
%    \lstset{language=XML}
%     \begin{lstlisting}
%
%<library path="lib/libslam_gmapping_nodelet">
%    <class name="SlamGMappingNodelet" type="SlamGMappingNodelet" base_class_type="nodelet::Nodelet">
%        <description>
%            Nodelet ROS wrapper for OpenSlams Gmapping.
%        </description>
%    </class>
%</library>
%      \end{lstlisting}
%
%\vspace{5mm}    Now run your node again.

\end{description}
\end{document}

