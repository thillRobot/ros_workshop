
% !TEX TS-program = xelatex
% !TEX encoding = UTF-8 Unicode

% Tennessee Technological University
% ME4140 - Fall 2016 - Fall 2017 - Fall 2020 - Fall 2021 
% Tristan Hill - September 10, 2020 - September 28, 2021
% Lecture 5 - Turtlebot Platform

\documentclass[fleqn]{beamer} % for presentation (has nav buttons at bottom)

%\usepackage{/home/thill/Documents/lectures/ros_workshop/ros_lecture}
\usepackage{/mnt/c/Users/thill/Documents/courses/ros_workshop/ros_lecture}

\newcommand{\MNUM}{5\hspace{2mm}} % Module number
\newcommand{\TNUM}{---\hspace{2mm}} % Topic number - single topic for now
\newcommand{\moduletitle}{Turtlebot Platform} % Titles and Stuff
%\newcommand{\topictitle}{---} 

\newcommand{\sectiontitleI}{A Brief History} % More Titles and Stuff
\newcommand{\sectiontitleII}{Recent Models}
\newcommand{\sectiontitleIII}{Turtlebot3 at TNTECH}
\newcommand{\sectiontitleIV}{Tutorial \MNUM - Turtlebot3}

\setbeamercolor{title in head/foot}{fg=TTUgold} % this needs work...

\author{ME4140 - ROS Workshop}
\title{Module \MNUM - \moduletitle}
\date{Mechanical Engineering\vspc Tennessee Technological University}


\begin{document}

\lstset{language=MATLAB,basicstyle=\ttfamily\small,showstringspaces=false}

\frame{\titlepage \center\begin{framed}\Large \textbf{Module \MNUM - \moduletitle}\end{framed} \vspace{5mm}}

% Section 0 - Outline
\frame{
	
	\large \textbf{Module \MNUM - \moduletitle} \vspace{3mm}\\
	
	\begin{itemize}
	
		\item \hyperlink{sectionI}{\sectiontitleI}    \vspc % Section I
		\item \hyperlink{sectionII}{\sectiontitleII} 	\vspc % Section II
		\item \hyperlink{sectionIII}{\sectiontitleIII} 	\vspc %Section III
		\item \hyperlink{sectionIV}{\sectiontitleIV} \vspc %Section IV	
	
	\end{itemize}

}

\section{\sectiontitleI}

	% Section I - Frame I
	\frame[containsverbatim]{ \small
		\frametitle{\sectiontitleI}



}


	% Section I - Frame II
	\frame[containsverbatim]{ \small
		\frametitle{\sectiontitleI}

	}

\section{\sectiontitleII}

	% Section II - Frame I
	\frame[containsverbatim]{ \small
		\frametitle{\sectiontitleII}
		
  \includegraphics[scale=.35]{turtlebot_family.png}


}

\section{\sectiontitleIII}

	% Section II - Frame I
	\frame{ \small
		\frametitle{\sectiontitleIII}
	\begin{multicols}{2}
		
	\includegraphics[scale=.35]{tb3.png}	
		
%	\begin{itemize}
%		\item On-board Computer - Rasp Pi
%		\item Drive Motors - Dynamixel Network Servos
%	  	\item Distance Sensor - 360 Laser Distance Sensor LDS-01
%	  	\item Camera - Raspberry Pi Camera Module v2.1
%	  	\item Inertial System - ()Gyro. 3 Axis, Accel. 3 Axis, Mag. 3 Axis)
%		
%	\end{itemize}	
		
		
		
	\end{multicols}

	{\tiny Full Details \href{https://emanual.robotis.com/docs/en/platform/turtlebot3/features/\#features}{here} }

}


\section{\sectiontitleIV}	
	% Section V - Frame I
	            \begin{frame}[label=sectionIV] \small
		\frametitle{\sectiontitleIV}    
	
 \setbeamertemplate{itemize items}[triangle]
                \begin{itemize}

					\item {\bf Overview:} Finally, you are going to install the Turtlebot3 simulator in ROS. This is a 3D robot simualator that uses Gazebo. 		

					\item {\bf Assignment:} Complete the Tutorial 5 Turtlebot3 Simulator on ilearn. You must be able to drive your turtlebot3 around the virtual arena.
                    
                    \item {\bf Deliverable:} Write a one to two paragraph summary of what you accomplished and what you struggled with the most. Include an image of the Gazebo window after you have driven the robot around with the keyboard. 
    
                    \item {\bf Next Week:} After completion of Module 5, you are ready to learn about robot navigation. You will learn to use the simulated Turtlebot3 to make a map a navigate in the Gazebo simulator. \vspc
                    
       
                \end{itemize}
		\end{frame}

\end{document}


%
%\begin{description}
%
%    \item [I. Components] The major building blocks of a ROS system
%
%        \begin{enumerate}
%                
%            \item \href{http://wiki.ros.org/Master}{Master Node}
%                \begin{itemize}
%                    \item {\it The ROS Master provides naming and registration services to the rest of the nodes in the ROS system.}** 
%                    \item master node runs first  {\fontfamily{qcr}\selectfont  \hspace{5mm} \$ roscore } \\
%                    \item core of the system or robot {\fontfamily{qcr}\selectfont  \hspace{5mm} \$ ROS\_MASTER\_URI=http://12345 } \\
%                    
%
%                \end{itemize}
%            
%            \item \href{http://wiki.ros.org/Nodes}{Nodes}             
%                \begin{itemize}
%                    \item {\it A node is a process that performs computation.}** 
%                    \item each 'program' or 'element' of the robot is a node\\examples:
%                        \begin{itemize} 
%                        \begin{multicols}{2}    
%                        
%                            \item sensor
%                            \item hardware driver    
%                            \item navigation 
%                            \item keyboard or joystick        
%                            
%                        \end{multicols}
%                        \end{itemize}
%                    \item start or run node individually after master\vspace{5mm}\\
%                    {\fontfamily{qcr}\selectfont  \hspace{5mm} \$ rosrun <packagename> <nodename> <options> } \vspace{1mm}\\
%                  
%                    \item all nodes are registered to the master and communicate in different ways
%                        \begin{itemize}
%                            \item  \href{http://wiki.ros.org/Topics}{topics} - publishing and subscribing
%                            \item \href{http://wiki.ros.org/Parameter%20Serverparameters}{parameter server} - static data
%                            \item \href{http://wiki.ros.org/Services}{services} - subroutine call
%                        \end{itemize}    
%                                      
%                \end{itemize}
%                
%            \item \href{http://wiki.ros.org/Packages}{Packages} 
%
%                \begin{itemize}
%                    \item {\it Software in ROS is organized in packages. A package might contain ROS nodes, a ROS-independent library, a dataset, configuration files, a third-party piece of software, or anything else that logically constitutes a useful module.}** 
%                    \item a collection of related nodes, each node belongs to a package
%                                     
%                    \item pre-built packages available with ros installation {\fontfamily{qcr}\selectfont  \hspace{5mm} -desktop-full}
%                    \item pre-built packages available for installation \\
%                    \begin{description}
%                        \item[apt] {\fontfamily{qcr}\selectfont  \hspace{1mm} \$ sudo apt-get install ros-<distribution>-<packagename> } 
%                        \item[rosdep]{\fontfamily{qcr}\selectfont  \hspace{1mm} \$ rosdep install <packagename> } \\
%                    \end{description}
%                    \item update ubuntu before installing anything \\
%                    {\fontfamily{qcr}\selectfont  \hspace{2mm} \$ sudo apt-get update } \\
%                    {\fontfamily{qcr}\selectfont  \hspace{2mm} \$ sudo apt-get check }                
%                \end{itemize}            
%
%                
%                
%    \end{enumerate}
%    ** from (ros.org)
%    \newpage
%    
%    \item [II. The Graph of the System] ROS works on a system of interconnected nodes. It is very useful to visualize this in a graph.\\
%        \begin{enumerate}   
%            
%            \item \href{http://wiki.ros.org/rqt_graph}{RQT Graph} A very useful tool. A node {\it rqt\_graph} in a package {\it rqt\_graph}. \\\\
%                    {\fontfamily{qcr}\selectfont  \hspace{5mm} \$ rosrun rqt\_graph rqt\_graph}\\
%            
%            \includegraphics[scale=.4]{ros_basics_fig1.png} \\
%            
%            \item \href{http://wiki.ros.org/rqt_plot}{RQT Plot} A very useful tool. A node {\it rqt\_plot} in a package {\it rqt\_plot}. \\
%                {\fontfamily{qcr}\selectfont  \hspace{5mm} \$ rosrun rqt\_plot rqt\_plot}\\
%            
%            \includegraphics[scale=.5]{ros_basics_fig2.png} \\    
%          \end{enumerate}       
%
%
%\newpage
%    
%    \item [III. Topics, Publishers, and Subscribers] The nodes in a ROS system communicate.  
%        \begin{enumerate}   
%           \item \href{http://wiki.ros.org/Topics}{Topics} 
%                    \begin{itemize}
%                        \item data available to nodes in the system
%                        \item each topic has a name    
%                        \item data is stored and transferred in standard ros data types
%                        \item generally data is streaming, but does not have to be
%                    \end{itemize}    
%           \item \href{http://wiki.ros.org/ROS/Tutorials/WritingPublisherSubscriber(c++)}{Publishers}
%                    \begin{itemize}
%      
%                        \item data produced by a node can be shared with the system by publishing a topic
%                        \item a node which outputs topic data is a publisher
%                        \item a node may publish multiple topics 
%                        
%                        \end{itemize}  
%            \item \href{http://wiki.ros.org/ROS/Tutorials/WritingPublisherSubscriber(c++)}{Subscribers}
%                \begin{itemize}
%      
%                        \item a registered node can access the data in a topic by subscribing to a topic
%                        \item a node which gets topic data as input is a subscriber
%                        \item a node may subscribe to multiple topics 
%                    \end{itemize}  
%                    
%            \item \href{http://wiki.ros.org/rostopic}{rostopic}                    
%                \begin{itemize}   
%                    \item a very useful tool, a built in package 
%                    \item used differently than other packages, does not require rosrun
%                    \item a set of different tools
%                    \begin{description}   
%                        \item [list] {\fontfamily{qcr}\selectfont  \hspace{5mm} \$ rostopic list}\\
%                        \item [echo] {\fontfamily{qcr}\selectfont  \hspace{5mm} \$ rostopic echo /topicname}\\
%                        \item [type] {\fontfamily{qcr}\selectfont  \hspace{5mm} \$ rostopic pub /topicname}\\
%                    \end{description}  
%                \end{itemize}    
%                    \item  \href{http://wiki.ros.org/msg}{data types} - topics are published in standard types called messages
%                    \begin{itemize}
%                        \item {\fontfamily{qcr}\selectfont  \hspace{5mm} std\textunderscore msgs/int32 }
%                        \item {\fontfamily{qcr}\selectfont  \hspace{5mm} std\textunderscore msgs/float32 }\\
%                        
%                        \item {\fontfamily{qcr}\selectfont  \hspace{5mm} geometry\textunderscore msgs/Point }
%                        \item {\fontfamily{qcr}\selectfont  \hspace{5mm} geometry\textunderscore msgs/Pose }\\
%                        
%                        \item {\fontfamily{qcr}\selectfont  \hspace{5mm} nav\textunderscore msgs/Odometry }
%                        \item {\fontfamily{qcr}\selectfont  \hspace{5mm} nav\textunderscore msgs/Path }\\
%                    \end{itemize}
%                    
%                    \item let show an example now!
%                                      
%        \end{enumerate}     
%\newpage
%    
%%    \item [IV. Services] The nodes in a ROS system can also communicate through services. Thsi is for reply/request type operations.  
%%  \begin{enumerate} 
%%\item \href{http://wiki.ros.org/Services}{Services}
%%\end{enumerate}    
%%\newpage
%    
%
%\end{description}
%\end{document}
%
