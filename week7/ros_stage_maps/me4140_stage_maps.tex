\documentclass[12pt]{article}
\usepackage{hyperref}
\usepackage[pdftex]{graphicx}
\usepackage{multirow}
\usepackage{setspace}
\usepackage{color}
\usepackage{multicol}
\usepackage{listings}
\usepackage{color}
\usepackage[T1]{fontenc}

\hypersetup{
    bookmarks=true,         % show bookmarks bar?
    unicode=false,          % non-Latin characters in Acrobat’s bookmarks
    pdftoolbar=true,        % show Acrobat’s toolbar?
    pdfmenubar=true,        % show Acrobat’s menu?
    pdffitwindow=false,     % window fit to page when opened
    pdfstartview={FitH},    % fits the width of the page to the window
    pdftitle={My title},    % title
    pdfauthor={Author},     % author
    pdfsubject={Subject},   % subject of the document
    pdfcreator={Creator},   % creator of the document
    pdfproducer={Producer}, % producer of the document
    pdfkeywords={keyword1, key2, key3}, % list of keywords
    pdfnewwindow=true,      % links in new PDF window
    colorlinks=true,       % false: boxed links; true: colored links
    linkcolor=red,          % color of internal links (change box color with linkbordercolor)
    citecolor=green,        % color of links to bibliography
    filecolor=magenta,      % color of file links
    urlcolor=blue           % color of external links
}


\lstset{frame=tb,
  language=Java,
  aboveskip=3mm,
  belowskip=3mm,
  showstringspaces=false,
  columns=flexible,
  basicstyle={\small\ttfamily},
  numbers=none,
  numberstyle=\tiny\color{gray},
  keywordstyle=\color{blue},
  commentstyle=\color{dkgreen},
  stringstyle=\color{mauve},
  breaklines=true,
  breakatwhitespace=true,
  tabsize=3
}

\definecolor{dkgreen}{rgb}{0,0.6,0}
\definecolor{gray}{rgb}{0.5,0.5,0.5}
\definecolor{mauve}{rgb}{0.58,0,0.82}

\definecolor{mygray}{rgb}{.6, .6, .6}
\definecolor{mypurple}{rgb}{0.6,0.1961,0.8}
\definecolor{mybrown}{rgb}{0.5451,0.2706,0.0745}
\definecolor{mygreen}{rgb}{0, .39, 0}

\newcommand{\R}{\color{red}}
\newcommand{\B}{\color{blue}}
\newcommand{\BR}{\color{mybrown}}
\newcommand{\K}{\color{black}}
\newcommand{\G}{\color{mygreen}}
\newcommand{\PR}{\color{mypurple}}

\newcommand{\pkgname}{\G<package\_name>\K}
\newcommand{\wspname}{\R<workspace\_name>\K}
\newcommand{\nodname}{\PR<node\_name>\K}
\newcommand{\usrname}{\B<user\_name>\K}
\newcommand{\tpcname}{/topic\_name}

\newcommand{\home}{\textasciitilde/}

\newcommand{\rosdistro}{kinetic}

\newcommand{\pthname}{/opt/ros/\rosdistro/share/turtlebot\_stage/maps/}



\textwidth=6.5in
\topmargin=-0.5in
\textheight=9.25in
\hoffset=-0.5in
\footskip=0.2in

\pagestyle{myheadings}
\markright{{\large ME 4140 Fall 2017---The Robotics Operating System}}



\usepackage{geometry}
 \geometry{
 a4paper,
 total={170mm,257mm},
 left=20mm,
 top=20mm,
 }

	

\begin{document}

\thispagestyle{plain}

\begin{center}
   {\bf \Large ROS - Modifying Your Map }\vspace{2mm} \\
   {\bf \large ME 4140 - Introduction to Robotics - Fall 2017} \\
\end{center}

This tutorial will guide you through the process of making your own map from the default map and modifying it for your own purposes. For more information read the \href{http://wiki.ros.org/turtlebot_stage/Tutorials/indigo/Customizing%20the%20Stage%20Simulator}{wiki}.\\

\begin{enumerate}
    \item Navigate to the install directory using $cd$. \vspace{4mm}\\   
        {\fontfamily{qcr}\selectfont
        \$ cd /opt/ros/\rosdistro/share/turtlebot\textunderscore stage/maps/} 
    \item Make a directory to put your maps in using {\it mkdir}.\vspace{4mm}\\
         {\fontfamily{qcr}\selectfont \$ mkdir \home\wspname/src/\pkgname/maps} 
    
    \item Copy 3 files into the directory you just made using $cp$. This is the directory that you want to keep them in.({\it not} with sudo).\\ \vspace{1mm}\\
        {\fontfamily{qcr}\selectfont    
        \$ cp maze.yaml \home\wspname/src/\pkgname/maps/new\_maze.yaml \\
        \$ cp maze.png \home\wspname/src/\pkgname/maps/new\_maze.png \\
        \$ cd stage \\
        \$ cp maze.world \home\wspname/src/\pkgname/maps/new\_maze.world }    
	\item Make two edits to the new\textunderscore maze.world file.\\
	{\fontfamily{qcr}\selectfont    
        \$ gedit \home\wspname/src/\pkgname/maps/new\_maze.world}\\
        {\it change:}
            {\fontfamily{qcr}\selectfont
			include "turtlebot.inc" } \\
		{\it to:}
            {\fontfamily{qcr}\selectfont
			\\include "/opt/ros/\rosdistro/share/turtlebot\_stage/maps/stage/turtlebot.inc" } \\
		{\it change:}
            {\fontfamily{qcr}\selectfont
			name "maze"
		  	bitmap "../maze.png" } \\
		{\it to:}\\
            {\fontfamily{qcr}\selectfont
			name "new\textunderscore maze"
		  	bitmap "new\textunderscore maze.png" }\\
	\item Make one edit to the new\textunderscore maze.yaml file\\
	{\fontfamily{qcr}\selectfont    
        \$ gedit \home\wspname/src/\pkgname/maps/new\_maze.yaml}

		{\it change:} 
            {\fontfamily{qcr}\selectfont
			image: maze.png } \\
		{\it to: }
            {\fontfamily{qcr}\selectfont
			image: new\textunderscore maze.png } 
	\item Edit your image file new\textunderscore maze.png with the image editor of your choice.\\
{\fontfamily{qcr}\selectfont \$ sudo apt-get install pinta }\\
    {\fontfamily{qcr}\selectfont \$ pinta \home\wspname/src/\pkgname/maps/new\textunderscore maze.png }
	\item Set your map as an environment variable and run the simulator
	
		{\fontfamily{qcr}\selectfont  
        \$ export TURTLEBOT\textunderscore STAGE\textunderscore MAP\textunderscore FILE=\\''/home/\usrname/\wspname/src/\pkgname/maps/new\textunderscore maze.yaml}"
        
		{\fontfamily{qcr}\selectfont 
        \$ export TURTLEBOT\textunderscore STAGE\textunderscore WORLD\textunderscore FILE=\\"/home/\usrname/\wspname/src/\pkgname/maps/new\textunderscore maze.world}"
        
    \item Launch a robot in a newly modified map !   \vspace{1mm}\\
		{\fontfamily{qcr}\selectfont   
        \$ roslaunch turtlebot\textunderscore stage turtlebot\textunderscore in\textunderscore stage.launch }

    

\newpage

 \item Now we want to use a map with a different size and resolution. To do this we need to make a few small changes.\\   
	\begin{enumerate}
		
		\item Set the name of the map in the .world file. If the map image is in a different directory include the entire path. Also, Set the map size and center in meters here. 
		
			       \begin{verbatim}
			floorplan
(
  name "brown_3rd"
  bitmap "brown_3rd.png"
  size [ 59.0 22.0 2.0 ]       #size of the map file in meters 
  pose [ 29.5 11.0 0.0 0.0 ]   #center of the map file in meters\\
)
\end{verbatim}
	\item Set the initial pose of the robot in the .world file.
		
			       \begin{verbatim}
turtlebot
(
  pose [ 5.0 9.0 0.0 0.0 ] #initial robot pose in meters
  name "turtlebot"
  color "black"
)
\end{verbatim}	  	
		\item Set the map name  in the .yaml file. Also, set the resolution in meters per pixel.
		\begin{verbatim}
		image: brown_3rd.png
resolution: 0.07 # map resolution in meters/pixel
origin: [0.0, 0.0, 0.0]
negate: 0
occupied_thresh: 0.65
free_thresh: 0.196
		\end{verbatim}	 
		
		\item Export the correct files. Also, you could put the export lines in your .bashrc file so you dont have to do the exports everytime.\\
		
		\item Turn on the simulator. Notice you also have to declare the initial pose here too. This could be fixed with a launch file which is covered in the next lesson. \\

{\fontfamily{qcr}\selectfont   
        \$ roslaunch turtlebot\textunderscore stage turtlebot\textunderscore in\textunderscore stage.launch initial\_pose\_x=5.0 initial\_pose\_y=9.0 initial\_pose\_a=0.0 } \\\\

	If that line causes an error try this one instead ( I am working on this bug!)\\

{\fontfamily{qcr}\selectfont  
 	\$ roslaunch turtlebot\textunderscore stage turtlebot\textunderscore in\textunderscore stage.launch initial\_pose\_x:=5.0 initial\_pose\_y:=9.0 }

	\end{enumerate}




\end{enumerate}


\end{document}

