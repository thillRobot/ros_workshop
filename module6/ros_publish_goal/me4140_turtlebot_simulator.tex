\documentclass[12pt]{article}
\usepackage{hyperref}
\usepackage[pdftex]{graphicx}
\usepackage{multirow}
\usepackage{setspace}
\usepackage{color}
\usepackage{multicol}
\usepackage{listings}
\usepackage{color}

\hypersetup{
    bookmarks=true,         % show bookmarks bar?
    unicode=false,          % non-Latin characters in Acrobat’s bookmarks
    pdftoolbar=true,        % show Acrobat’s toolbar?
    pdfmenubar=true,        % show Acrobat’s menu?
    pdffitwindow=false,     % window fit to page when opened
    pdfstartview={FitH},    % fits the width of the page to the window
    pdftitle={My title},    % title
    pdfauthor={Author},     % author
    pdfsubject={Subject},   % subject of the document
    pdfcreator={Creator},   % creator of the document
    pdfproducer={Producer}, % producer of the document
    pdfkeywords={keyword1, key2, key3}, % list of keywords
    pdfnewwindow=true,      % links in new PDF window
    colorlinks=true,       % false: boxed links; true: colored links
    linkcolor=red,          % color of internal links (change box color with linkbordercolor)
    citecolor=green,        % color of links to bibliography
    filecolor=magenta,      % color of file links
    urlcolor=blue           % color of external links
}

\definecolor{dkgreen}{rgb}{0,0.6,0}
\definecolor{gray}{rgb}{0.5,0.5,0.5}
\definecolor{mauve}{rgb}{0.58,0,0.82}
\lstset{frame=tb,
  language=Java,
  aboveskip=3mm,
  belowskip=3mm,
  showstringspaces=false,
  columns=flexible,
  basicstyle={\small\ttfamily},
  numbers=none,
  numberstyle=\tiny\color{gray},
  keywordstyle=\color{blue},
  commentstyle=\color{dkgreen},
  stringstyle=\color{mauve},
  breaklines=true,
  breakatwhitespace=true,
  tabsize=3
}

% ME4140 - Fall 2016

\textwidth=6.5in
\topmargin=-0.5in
\textheight=9.25in
\hoffset=-0.5in
\footskip=0.2in

\pagestyle{myheadings}
\markright{{\large ME 4140 Fall 2016---The Robotic Operating System}}



\newcommand{\pkgname}{package\textunderscore name}
\newcommand{\wspname}{workspace\textunderscore name}
\newcommand{\nodname}{node\textunderscore name}
\newcommand{\tpcname}{topic\textunderscore name}
\newcommand{\lfname}{file\textunderscore name}

\newcommand{\pthname}{/opt/ros/indigo/share/turtlebot\textunderscore stage/maps/}


\begin{document}

\thispagestyle{plain}

\begin{center}
   {\bf \Large ROS - Publishing and Subscribing to The Turtlebot Simulator}\vspace{2mm} \\
   {\bf \large ME 4140 - Introduction to Robotics - Fall 2016} \\
\end{center}


\begin{enumerate}
	
	\item This tutorial assumes you have been following the course so far. TO begin create a new package with the name of your choosing.


		{\fontfamily{qcr}\selectfont  \hspace{5mm} \$ catkin\textunderscore create\textunderscore pkg publish\textunderscore goal std\textunderscore msgs rospy roscpp }
    

	\item Open a new file in the proper src folder and insert the following code. 

     	\begin{lstlisting}

	#include "ros/ros.h"
#include "geometry_msgs/Twist.h"
#include "geometry_msgs/PoseStamped.h"
#include <sstream>
int main(int argc, char **argv)
{
	ros::init(argc, argv, "publish_goal");
	ros::NodeHandle n;
	ros::Publisher ttu_publisher =
	n.advertise<geometry_msgs::PoseStamped>("/move_base_simple/goal", 1000);
	ros::Rate loop_rate(10);

	geometry_msgs::PoseStamped msg;
	msg.header.stamp=ros::Time::now();
	msg.header.frame_id="map";

	int count = 0;
	while ((ros::ok())&&(count<5))  
	{				
		msg.pose.position.x = 3.0;
		msg.pose.position.y = 2.0;
		msg.pose.position.z = 0;

		msg.pose.orientation.w = 1.0;

		ttu_publisher.publish(msg);
		ros::spinOnce();
		loop_rate.sleep();
		++count;
	}
}


\end{lstlisting}
	
    \item Now you need to install the 'turtlebot' simulator into your ROS system.
    
	{\fontfamily{qcr}\selectfont  \hspace{5mm} \$ sudo apt-get install ros-indigo-turtlebot-simulator }
	
    \item Now install the physical 'turtlebot' drivers into your ROS system. This step may only be necessary if you are using a real robot. \href{http://wiki.ros.org/turtlebot/Tutorials/indigo/Turtlebot%20Installation} {Link Here} \\
    
    {\fontfamily{qcr}\selectfont  \hspace{5mm} \$ sudo apt-get install ros-indigo-turtlebot ros-indigo-turtlebot-apps ros-indigo-turtlebot-interactions ros-indigo-turtlebot-simulator ros-indigo-kobuki-ftdi ros-indigo-rocon-remocon ros-indigo-rocon-qt-library ros-indigo-ar-track-alvar-msgs}
    
    \item This simulates a physical robot in a 2D world. Next we neeed to setup the world. There are 3 important files that control the world. Your installation came with a demo world.
    \begin{itemize}
    
        \item {\fontfamily{qcr}\selectfont  \hspace{5mm} \pthname maze.png}
        \item {\fontfamily{qcr}\selectfont  \hspace{5mm} \pthname maze.yaml}
        \item {\fontfamily{qcr}\selectfont  \hspace{5mm} \pthname stage/maze.world}
    
    \end{itemize}

    \item First try the simulator in the demo world called {\it maze}. We will export the files as {\it environment variables}

    {\fontfamily{qcr}\selectfont  \hspace{5mm} \$ export TURTLEBOT\textunderscore STAGE\textunderscore MAP\textunderscore FILE=\\"\pthname maze.yaml"}\\
 
    {\fontfamily{qcr}\selectfont  \hspace{5mm} \$ export TURTLEBOT\textunderscore STAGE\textunderscore WORLD\textunderscore FILE=\\"\pthname stage/maze.world"}\\
    
    \item Now use the launch file (available upon install) to start the simulator.\\
    {\fontfamily{qcr}\selectfont  \hspace{5mm} \$ roslaunch turtlebot\textunderscore stage turtlebot\textunderscore in\textunderscore stage.launch}
    
    \item Now you can modify the world you have just simulated. To do this copy all three files and rename them something sensible. Open the {\it .png} file with any image editor, and draw on it and save. You also need to modify just a few lines in the {\it .yaml} file and the {\it .world} file. (Note: This step will be detailed in the next tutorial. Continue at your own risk or contact me for help.)
    
    
\end{enumerate}
\end{document}

